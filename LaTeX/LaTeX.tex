\documentclass{article}

\title{\LaTeX}
\author{zz-sj}

\usepackage{CJKutf8}


\begin{document}
\maketitle
\begin{CJK}{UTF8}{gbsn}
%\tableofcontents

\section{基础概念}
LaTex 的源码结构分为导言区和正文区。

\subsection{导言区}
导言区用于指定文章类型、设置文章title以及导入文章需要使用的包等。具体命令有:

$\backslash$documentclass\{...\}用来指定文章类型,常用参数有article、book、report等

$\backslash$title\{...\}设置标题内容,$\backslash$author\{...\}设置文章作者名称,$\backslash$date\{...\}设置文章时间。设置好后,需要在正文区调用$\backslash$maketitle来显示标题。

$\backslash$usepackage\{...\}导入文章使用到的命令包。

$\backslash$newcommand\{...\}定义新命令。

\subsection{正文区}
正文区用于书写文章内容,其基本结构为:

$\backslash$begin\{document\}

	\ \ \ \ \ \  \%正文内容

$\backslash$end\{document\}

正文中使用\$...\$括起来的内容显示为数学公式体,\$\$...\$\$为换行数学公式体。

\% 可以用来注释单行内容。

\section{字体设置}
\subsection{字体族的设置(罗马字体、打印机字体等)}
\ \ \ \  $\backslash$textrm\{...\}来设置字体族为罗马字体 \textrm{Roman Family}

$\backslash$textsf\{...\}来设置字体族为无衬线字体 \textsf{Sans Serif Family}

$\backslash$texttt\{...\}来设置字体族为打字机字体 \texttt{Typewriter Family}   \ \ 等等。

也可以使用命令来声明从目前行开始,往后的文字的字体,使用\{...\}来限定命令的作用域,比如:

\{$\backslash$rmfamily ...\}\{$\backslash$sffamily ...\}\{$\backslash$ttfamily ...\}等等。

\subsection{字体系列的设置(粗细,宽度)}
\ \ \ \ $\backslash$textmd\{...\}来设置字体系列为中等粗细 \textmd{Medium Series}

$\backslash$textbf\{...\}来设置字体系列为粗体字体 \textbf{Boldface Series }
  \ \ 等等。

也可以使用命令来声明从目前行开始,往后的文字的系列,使用\{...\}来限定命令的作用域,比如:

\{$\backslash$mdseries ...\}\{$\backslash$bfseries ...\} 等等。

\subsection{字体形状的设置(直立、斜体、伪斜体、小型大写)}
字体形状

$\backslash$textup\{...\}来设置字体族为直立 \textup{Upright Shape}

$\backslash$textit\{...\}来设置字体族为斜体 \textit{Italic Shape}

$\backslash$textsl\{...\}来设置字体族为伪斜体 \textit{Slanted Shape}

$\backslash$textsc\{...\}来设置字体族为小型大写 \textsc{Small Caps Shape}
等等。

也可以使用命令来声明从目前行开始,往后的文字的字体,使用\{...\}来限定命令的作用域,比如:

\{$\backslash$upshape ...\}\{$\backslash$itshape ...\}\{$\backslash$slshape ...\}\{$\backslash$scshape ...\} 等等。
\\
字体大小

\{$\backslash$tiny ...\}  {\tiny Hello}

\{$\backslash$scriptsize ...\}  {\scriptsize Hello}

\{$\backslash$footnotesize ...\}  {\footnotesize Hello}

\{$\backslash$small ...\}  {\small Hello}

\{$\backslash$normalsize ...\}  {\normalsize Hello}

\{$\backslash$large ...\}  {\large Hello}

\{$\backslash$Large ...\}  {\Large Hello}

\{$\backslash$LARGE ...\}  {\LARGE Hello}
s
\{$\backslash$huge ...\}  {\huge Hello}

\{$\backslash$Huge ...\}  {\Huge Hello}
\\
\section{文章的结构}
\ \ \ $\backslash$section\{本小节名\}来将接下来的文字设置为一个小节,直到遇到下一个$\backslash$section\{...\}。在小节中,可以使用$\backslash$subsection\{子小节名\}来设置子小节,使用$\backslash$subsubsection\{子子小节名\}来设置子小节的子小节。

$\backslash$$\backslash$命令会使得文字换行,但不会产生新的段落,所以换行后也不会产生缩进。

$\backslash$paragraph 命令会产生新的段落,产生换行和缩进。

$\backslash$tableofcontents命令产生文档目录。

\section{文章的插图}
导言区: $\backslash$usepackage\{graphicx\}

\ \ \ \ \ \ \ $\backslash$graphicspath\{\{path1\}, \{path2\}\}
\\
语 法:  \ \ $\backslash$includegraphics[ options ] \{filename\}\\
 选项中的参数有:scale=0.3缩放因子 height=2cm固定值的图像高度 width固定值的图像宽度
 angle旋转角度 
 
 \section{表格的制作}

		

\end{CJK}

\end{document}