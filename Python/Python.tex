\documentclass{book}

\title{Python}
\author{zz}

\usepackage{CJKutf8}


\begin{document}
\maketitle
\pagestyle{empty}
\begin{CJK}{UTF8}{gbsn}
\tableofcontents
\renewcommand\thesection{\arabic{section}} 

\newpage
\section{Python基础}


%\subsubsection{逻辑操作}
%\noindent$>$	\ \ \ 	大于
%
%\noindent$>=$	\ 	大于等于 
%
%\noindent$<$	\ \ \ 	小于
%
%\noindent$<=$	\ 	小于等于
%
%\noindent!=		\ \ \ 不等于
%
%\noindent in	\ \ \ \ 	判断特定的值是否已包含在列表中
%
%\noindent not in	判断特定的值是否未包含在列表中
%\subsubsection{逻辑连接}
%\noindent and		逻辑与,表达式全为真时整个表达式为真
%
%\noindent or	\ \ 	逻辑或,表达式任一个为真则为真


\subsection{程序输入}
input\_var = input("Please input something:")

函数 input() 让程序暂停运行,等待用户输入一些文本。获取用户输入后,Python将其存储在一个变量中。函数 input() 接受一个参数:即要向用户显示的``提示''或者``说明'',让用户知道该怎么做。可以将参数先存储在变量中,然后将变量传递给 input()。

使用函数 input() 时,Python将用户输入解读为``字符串''。因此输入整数、浮点数等时需要强制格式转换。

\subsection{条件语句}

\subsubsection{if语句}
if  boolean\_exp :

​	exp;

​	if-elif-else结构依次检查每个测试条件,直到遇到通过了的条件,Python将执行紧跟在其后边的代码块,并跳过余下的代码块。else是一条包罗万象的语句,只要不满足任何 if 和 elif 的条件测试,其中的代码块就会被执行,这可能会引入无效甚至恶意的数据。如果知道最终要测试的条件,应考虑使用一个 elif 来代替 else 。这样你可以肯定仅当满足相应的条件时,你的代码才会被执行。

在 if 语句中将列表名用在条件表达式中时,Python将在列表至少包含一个元素时返回True,并在列表为空时返回False。


\subsection{循环语句}
\subsubsection{for循环}
for value in iterator:

	use value do something
	
for循环可以遍历任何可迭代对象iterator中的所有元素。每次返回一个元素赋值给for关键字后的变量名value。
\subsubsection{while循环}
while boolean\_exp:

	do something
	
while循环当布尔表达式为真时一直运行。

在要求很多条件都满足时才继续运行的程序中,使用一个变量作为标志,用于判断整个程序是否处于活动状态是一个不错的做法。任何一个事件导致标志位False都将结束循环。
\subsubsection{break}

break语句跳出当前整个循环,不在继续执行余下的循环轮次。
\subsubsection{continue}
\ \ \ \ continue语句跳出当前循环轮次,回到循环的开头,判断循环条件后进行下一次循环。

for循环是一种遍历列表的有效方式,但是for循环中不应该修改列表,否则将导致Python难以跟踪其中的元素。要在遍历列表的同时对其进行修改,可以使用while循环。通过while循环同列表和字典结合使用,可以收集、存储并组织大量的输入。


\subsection{Python注释}
\subsection{Python关键字}
\subsection{Python运算符}



\section{String}


\subsection{Python字符串}
\ \ \ \ 字符串是 Python 中最常用的数据类型。我们可以使用引号('或")来创建字符串。
Python 不支持单字符类型,单字符在 Python 中也是作为一个字符串使用。Python 访问子字符串时,可以使用方括号来截取字符串中的字符或子字符串。

Python使用加号(+)来拼接字符串。

函数 str() 可以将非字符串的值转换为字符串。

\subsection{Python转义字符}
\begin{center}
\begin{tabular}{|l|l|}
	\hline 转义字符   					& 描述\\ 	
	\hline $\backslash$(在行尾时) 		& 续行符\\
	\hline $\backslash$$\backslash$		& 反斜杠符号\\
	\hline $\backslash$$'$		        & 单引号\\
	\hline $\backslash$$''$				& 双引号\\
	\hline $\backslash$b					& 退格(Backspace)\\
	\hline $\backslash$e					& 转义\\
	\hline $\backslash$000				& 空\\	
	\hline $\backslash$n					& 换行\\
	\hline $\backslash$v					& 纵向制表符\\
	\hline $\backslash$t					& 横向制表符\\
	\hline $\backslash$r     			& 回车     \\
	\hline $\backslash$f      			& 换页    \\
	\hline
\end{tabular}
\end{center}

\subsection{Python字符串运算符}
下表实例变量 a 值为字符串 ``Hello",b 变量值为 ``Python"
\begin{center}
\begin{tabular}{|l|l|l|}
	\hline 操作符 & 描述 & 实例\\ 	
	\hline + 	 & 字符串连接 &  $>>>$a + b   \ \ \ \ \ 'HelloPython' \\
	\hline *	 & 重复输出字符串 & $>>>$a * 2 \ \ \ \ \ \  'HelloHello' \\
	\hline [\ ]	 & 通过索引获取字符串中字符 & $>>>$a[1] \ \ \ \ \ \ \ \ 'e'\\
	\hline [ : ] & 双引号 & $>>>$a[1:4] \ \ \ \ \ \ 'ell' \\
	\hline in    & 如果字符串中包含给定的字符返回True & $>>>$'H' in a \ 															\ \ True\\
	\hline not in& 如果字符串中不包含给定的字符返回True & $>>>$'H' not in a \ \ False\\
	\hline r/R	 & 所有的字符串都是直接按照字面的意思来使用 & $>>>$print r'$\backslash$n'	 \ \ $\backslash$n	 \\	
	\hline \% 	 & 格式字符串 &   \\
	\hline
\end{tabular}
\end{center}


\subsection{Python三引号}
Python 中三引号可以将复杂的字符串进行赋值。

Python 三引号允许一个字符串跨多行,字符串中可以包含换行符、制表符以及其他特殊字符。

三引号的语法是一对连续的单引号或者双引号(通常都是成对的用)。

三引号让程序员从引号和特殊字符串的泥潭里面解脱出来,自始至终保持一小块字符串的格式是所谓的WYSIWYG(所见即所得)格式的。一个典型的用例是,当你需要一块HTML或者SQL时,这时当用三引号标记,使用传统的转义字符体系将十分费神。

\subsection{字符串内建函数}
\textbf{1. str.capitalize()} \newline
功能:\par 返回一个当前整个字符串的第一个字母大写的版本,原字符串不变。\newline 
参数:\par 无
\newline

\noindent \textbf{2. str.casefold()} \newline
功能:\par 返回一个当前整个字符串的全小写的版本,原字符串不变。与str.lower()的区别为后者只能用于ASCII码字符串,前者可以用于多国语言。\newline
参数:\par 无
\newline

\noindent \textbf{3. str.center(width[, fillchar])} \newline
功能:\par 返回一个指定的宽度 width 使当前字符串居中的新字符串,fillchar 为填充的字符,默认为空格。如果 width 小于字符串宽度直接返回字符串,不会截断。fillchar 只能是单个字符,默认为空格。\newline
参数:\par width --- 新字符串的宽度
     \par fillchar --- 填充字符
\newline

\noindent \textbf{4. str.count(sub[, start[, end]])} \newline
功能:\par 返回子字符串在字符串中出现的次数。方法用于统计字符串里某个字符出现的次数。可选参数为在字符串搜索的开始与结束位置。\newline
参数:\par sub --- 搜索的子字符串
  	 \par start --- 字符串开始搜索的位置。默认为第一个字符,索引值为0。
  	 \par end --- 字符串结束搜索的位置。默认为字符串的最后一个位置。
\newline

\newpage       
\noindent \textbf{5. str.encode(encoding='utf-8', errors='strict')} \newline
功能:\par 返回编码后的字符串。该方法以 encoding 指定的编码格式编码字符串。errors
参数可以指定不同的错误处理方案。\newline
参数:\par encoding --- 要使用的编码,默认为"UTF-8"。
     \par errors --- 设置不同错误的处理方案,默认为 'strict',意为编码错误引起一个UnicodeError。 其他可能的值有 'ignore', 'replace', 'xmlcharrefreplace', 'backslashreplace' 以及通过 codecs.register\_error() 注册的任何值。
\newline
 
\noindent \textbf{6. str.endswith(suffix[, start[, end]])} \newline
功能:\par 返回一个bool值。方法用于判断字符串是否以指定后缀结尾,如果以指定后缀结尾返回True,否则返回 False。可选参数 "start" 与 "end" 为检索字符串的开始与结束位置。
\newline
参数:\par suffix --- 被匹配的字符串。该参数可以是一个字符串或者是几个字符串组成的元组。
     \par start --- 字符串中的开始位置。
     \par end --- 字符中结束位置。     
\newline

\noindent \textbf{7. str.expandtabs(tabsize=8)} \newline
功能:\par 返回字符串中的 tab 符号('$\backslash$t')转为空格后生成的新字符串。方法把字符串中的 tab 符号('$\backslash$t')转为空格,tab 符号('$\backslash$t')默认的空       格数是 8。添加的空格数为一个制表符大小减去已有的字符个数。\newline
参数:\par tabsize -- 指定转换字符串中的 tab 符号('$\backslash$t')转为空格的字符数。          
\newline

\noindent \textbf{8. str.find(sub[, start[, end]])} \newline
功能:\par 返回包含的子字符串开始的索引值,如果不包含子字符串返回-1。\newline
参数:\par sub -- 指定检索的字符串
     \par start -- 开始的索引,默认为0。
     \par end -- 结束的索引,默认为字符串的长度。
\newline

\noindent \textbf{9. str.format()} \newline
功能:\par 字符串的格式化。详情见字符串格式化。\newline
参数:\par 无。
\newline

\newpage
\noindent \textbf{10. str.format\_map(mapping)} \newline
功能:\par 字符串的格式化。详情见字符串格式化。\newline
参数:\par mapping -- 格式化字符串需要的一个字典对象。
\newline

\noindent \textbf{11. str.index(sub, start=None, end=None)} \newline
功能:\par 如果包含子字符串,则返回开始的索引值,否则抛出异常ValueError。(和find()类似,但是会抛出异常)。\newline
参数:\par sub -- 指定检索的字符串。
     \par start -- 开始的索引,默认为0。
     \par end -- 结束的索引,默认为字符串的长度。
\newline


\noindent \textbf{12. str.isalnum()} \newline
功能:\par 如果 string 至少有一个字符并且所有字符都是[字母或数字]则返回 True,否则返回 False。\newline
参数:\par 无。
\newline

\noindent \textbf{13. str.isalpha()} \newline
功能:\par 如果字符串至少有一个字符并且所有字符都是字母则返回 True,否则返回 False。\newline
参数:\par 无。
\newline

\noindent \textbf{14. str.isdecimal()} \newline
功能:\par 检查字符串是否只包含十进制字符。这种方法只存在于unicode对象。注意:定义一个十进制字符串,只需要在字符串前添加 'u' 前缀即可。\newline
参数:\par 无。
\newline

\noindent \textbf{15. str.isdigit()} \newline
功能:\par 如果字符串只包含数字则返回 True 否则返回 False。\newline
参数:\par 无。
\newline

\newpage
\noindent \textbf{16. str.isidentifier()} \newline
功能:\par 如果字符串是有效的 Python 标识符返回 True,否则返回 False。可用来判断变量名是否合法。\newline
参数:\par 无。
\newline

\noindent \textbf{17. str.islower()} \newline
功能:\par 如果字符串中包含至少一个区分大小写的字符,并且所有这些(区分大小写的)字符都是小写,则返回 True,否则返回 False。\newline
参数:\par 无。
\newline

\noindent \textbf{18. str.isnumeric()} \newline
功能:\par 如果字符串中只包含数字字符,则返回 True,否则返回 False。这种方法是只针对unicode对象。\newline
参数:\par 无。
\newline

\noindent \textbf{19. str.isprintable()} \newline
功能:\par 如果字符串中为空或者字符串中所有的字符都是可打印字符,则返回 True,否则返回 False。\newline
参数:\par 无。
\newline

\noindent \textbf{20. str.isspace()} \newline
功能:\par 如果字符串中只包含空格,则返回 True,否则返回 False。\newline
参数:\par 无。
\newline

\noindent \textbf{21. str.istitle()} \newline
功能:\par 如果字符串中所有的单词拼写首字母是否为大写,且其他字母为小写则返回 True,否则返回 False。\newline
参数:\par 无。
\newline

\noindent \textbf{22. str.isupper()} \newline
功能:\par 如果字符串中包含至少一个区分大小写的字符,并且所有这些(区分大小写的)字符都是大写,则返回 True,否则返回 False。\newline
参数:无。\par 

\noindent \textbf{23. str.join(iterable)} \newline
功能:\par 用于将序列中的元素以指定的字符(str)连接生成一个新的字符串。\newline
参数:\par iterable --- 要连接的元素序列。
\newline

\noindent \textbf{24. str.ljust(width, fillchar=None)} \newline
功能:\par 返回一个原字符串左对齐,并使用空格填充至指定长度的新字符串。如果指定的长度小于原字符串的长度则返回原字符串。\newline
参数:\par width --- 指定字符串长度。
	 \par fillchar -- 填充字符,默认为空格。
\newline

\noindent \textbf{25. str.lower()} \newline
功能:\par 返回将字符串中所有大写字符转换为小写后生成的字符串拷贝。\newline
参数:\par 无。
\newline

\noindent \textbf{26. str.lstrip(chars=None)} \newline
功能:\par 返回截掉字符串左边的空格或指定字符后生成的新字符串。\newline
参数:\par chars --指定截取的字符,默认为空格。
\newline

\noindent \textbf{27. str.maketrans(*args, **kwargs)} \newline
功能:\par   功能:maketrans() 方法用于给 translate() 方法创建字符映射转换表。可以只接受一个参数,此时这个参数是个字典类型(暂不研究)。对于接受两个参数的最简单的调用方式,第一个参数是字符串,表示需要转换的字符,第二个参数也是字符串,表示转换的目标。两个字符串的长度必须相同,为一一对应的关系。在Python3中可以有第三个参数,表示要删除的字符,也是字符串。一般 maketrans() 方法需要配合translate() 方法一起使用。\newline
参数:\par intab -- 需要转换的字符组成的字符串。
     \par outtab -- 转换的目标字符组成的字符串。
     \par delchars -- 可选参数,表示要删除的字符组成的字符串。\newline
示例: \par intab = "aeiou"
     \par outtab = "12345"
     \par deltab = "thw"
 	 \par trantab1 = str.maketrans(intab,outtab)  
	 \par trantab2 = str.maketrans(intab,outtab,deltab)  
	 \par test = "this is string example....wow!!!"
	 \par print(test.translate(trantab1))
	 \par print(test.translate(trantab2))
     \par 输出:
     \par th3s 3s str3ng 2x1mpl2....w4w!!!
	 \par 3s 3s sr3ng 2x1mpl2....4!!!
\newline

\noindent \textbf{28. str.partition(sep)} \newline
功能:\par 方法用来根据指定的分隔符将字符串进行分割。返回一个3元的元组,第一个为分隔符左边的子串,第二个为分隔符本身,第三个为分隔符右边的子串。\newline
参数:\par sep --- 指定的分割符。
\newline

\noindent \textbf{29. str.replace(old, new, count=None)} \newline
功能:\par 返回字符串中的 old(旧字符串)替换成 new(新字符串)后生成的新字符串,如果指定第三个参数count,则替换不超过count次。\newline
参数:\par old --- 将被替换的子字符串。
     \par new --- 新字符串,用于替换old子字符串。
     \par count --- 可选字符串, 替换不超过 count 次。
\newline

\noindent \textbf{30. str.rfind(sub, start=None, end=None)} \newline
功能:\par 返回字符串最后一次出现的位置(从右向左查询)如果没有匹配项则返回
-1。\newline
参数:\par str --- 查找的字符串。
     \par start --- 开始查找位置,默认为首字符,下标为0。
     \par end --- 结束查找位置,默认为字符串的长度。
\newline

\noindent \textbf{31. str.rindex(sub, start=None, end=None)} \newline
功能:\par 返回字符串最后一次出现的位置(从右向左查询),如果没有匹配项则抛出异常ValueError。\newline
参数:\par str --- 查找的字符串。
     \par start --- 开始查找位置,默认为首字符,下标为0。
     \par end --- 结束查找位置,默认为字符串的长度。
\newline

\noindent \textbf{32、str.rjust(width, fillchar=None)} \newline
功能:\par 返回一个原字符串右对齐,并使用空格填充至长度 width 的新字符串。如果指定的长度小于字符串的长度则返回原字符串。\newline
参数:\par width --- 指定填充指定字符后中字符串的总长度.
	 \par fillchar --- 填充的字符,默认为空格。
\newline

\noindent \textbf{33、str.rpartition(sep)} \newline
功能:\par 方法从右边开始查找,根据指定的分隔符将字符串进行分割。返回一个3元的元组,第一个为分隔符左边的子串,第二个为分隔符本身,第三个为分隔符右边的子串。\newline
参数:\par sep --- 指定的分割符。
\newline

\noindent \textbf{34、str.rsplit(sep=None, maxsplit=-1)} \newline
功能:\par 通过指定分隔符对字符串从右边开始进行切片,如果参数 num 有指定值,则分隔成为 num+1 个子字符串。返回分割后的字符串列表。\newline
参数:\par sep --- 指定的分割符。
     \par maxsplit --- 分割次数。
\newline

\noindent \textbf{35. str.rstrip(chars=None)} \newline
功能:\par 返回截掉字符串右边的空格或指定字符后生成的新字符串。\newline
参数:\par chars --- 指定截取的字符,默认为空格。
\newline

\noindent \textbf{36. str.split()} \newline
功能:\par 通过指定分隔符对字符串从左边开始进行切片,如果参数 num 有指定值,则分隔成为 num+1 个子字符串。返回分割后的字符串列表。\newline
参数:\par sep --- 指定的分割符。
     \par maxsplit --- 分割次数。
\newline

\noindent \textbf{37. str.splitlines(keepends=None)} \newline
功能:\par 按照行('$\backslash$r', '$\backslash$r$\backslash$n', '$\backslash$n')分隔,返回一个包含各行作为元素的列表,如果参数 keepends 为 False,不包含换行符,如果为 True,则保留换行符。\newline
参数:\par keepends -- 在输出结果里是否保留换行符,默认为 False。
\newline

\noindent \textbf{38. str.startswith(prefix, start=None, end=None)} \newline
功能:\par 方法用于检查字符串是否是以指定子字符串开头,如果是则返回 True,否则返回 False。如果参数 start 和 end 指定值,则在指定范围内检查。\newline
参数:\par prefix -- 检测的字符串。
	 \par start -- 可选参数用于设置字符串检测的起始位置。
	 \par end -- 可选参数用于设置字符串检测的结束位置。
\newline
 
\noindent \textbf{39. str.strip(chars=None)} \newline
功能:\par 方法用于移除字符串头尾指定的字符(默认为空格或换行符)或字符序列。注意:该方法只能删除开头或是结尾的字符,不能删除中间部分的字符。\newline
参数:\par chars -- 移除字符串头尾指定的字符序列。
\newline

\noindent \textbf{40. str.swapcase()} \newline
功能:\par 方法用于对字符串的大小写字母进行转换。返回大小写字母转换后生成的新字符串。\newline
参数:\par 无。
\newline

\noindent \textbf{41. str.title()} \newline
功能:\par 方法返回"标题化"的字符串,就是说所有单词都是以大写开始,其余字母均为小写。\newline
参数:\par 无。
\newline

\noindent \textbf{42. str.translate(table)} \newline
功能:\par 方法根据参数table给出的表(包含 256 个字符)转换字符串的字符。\newline
参数:\par table -- 翻译表,翻译表是通过maketrans方法转换而来。
\newline

\noindent \textbf{43. str.upper()} \newline
功能:\par 方法将字符串中的小写字母转为大写字母。返回小写字母转为大写字母的字符串。\newline
参数:\par 无。
\newline

\noindent \textbf{45. str.zfill(width)} \newline
功能:\par 方法返回指定长度的字符串,原字符串右对齐,前面填充0。\newline
参数:\par width --- 指定字符串的长度。
\newline


\end{CJK}

\end{document}